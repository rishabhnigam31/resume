%______________________________________________________________________________________________________________________
% @brief    LaTeX2e Resume for Kamil K Wojcicki
\documentclass[margin,line]{resume}
\usepackage{hyperref}

%______________________________________________________________________________________________________________________
\begin{document}
\name{\Large Rishabh Nigam}
\begin{resume}

    %__________________________________________________________________________________________________________________
    % Contact Information
    \section{\mysidestyle Contact\\Information}

    Dept. of Computer Science and Engineering                            \hfill e-mail: rishabhn@cse.iitk.ac.in  \vspace{0mm}\\\vspace{0mm}%
    Indian Institute of Technology, Kanpur                         \hfill Mobile: +91 9005510276 \vspace{0mm}\       \vspace{0mm}\\\vspace{-4.5mm}%


    %__________________________________________________________________________________________________________________
    % Research Interests
    \section{\mysidestyle Research\\Interests}

    \begin{itemize}\itemsep0pt
\item Design and Analysis of Algorithms
\item Competitive Programming
\item Computational Complexity
\item Machine Learning and Artificial Intelligence
\item Programming Languages

\end{itemize}


    %__________________________________________________________________________________________________________________
    % Education
    \section{\mysidestyle Education}

    \textbf{Indian institute of Technology, Kanpur, India} \vspace{2mm}\\\vspace{1mm}%
    \textsl{B.Tech in Computer Science and Engineering} \hfill \textbf{ July 2010 -- present}\vspace{-3mm}\\\vspace{-1mm}%
    \begin{list2}
    	\item Cumulative Performance Index (CPI) of \textbf{\textsf{9.0}} (on a scale of 10)
        \item Expected graduation date: July 2014
    \end{list2}\vspace{-1.5mm}
    \textbf{SunBeam Academy, Varanasi} \vspace{2mm}\\\vspace{1mm}%
    \textsl{All India Senior Secondary Certificate Examination (AISSCE)} \hfill \textbf{ July 2008 -- May 2010}\vspace{-3mm}\\\vspace{-1mm}%
    \begin{list2}
    	\item Scored \textbf{\textsf{92.4\%}} marks in AISSCE Examination 
    \end{list2}
    \textbf{SunBeam Academy, Varanasi} \vspace{2mm}\\\vspace{1mm}%
    \textsl{All India Secondary School Examination (AISSE)} \hfill \textbf{ July 2006 -- May 2008}\vspace{-3mm}\\\vspace{-1mm}%
    \begin{list2}
    	\item Scored \textbf{\textsf {92.0\%}} marks in AISSE Examination 
    \end{list2}
    \vspace{-1mm}%


    %__________________________________________________________________________________________________________________
    % Honours and Awards
    \section{\mysidestyle Scholastic Achievements} 
    \begin{itemize}
    \item Secured \textbf{\textsf{All India Rank 39}} (among 4,00,000 students) in IIT-JEE 2010.
    \item Secured \textbf{\textsf{All India rank 10}} (among 11,00,000 students) in AIEEE 2010.
    \item Achieved \textbf{\textsf{All India Rank 1}} in UPSEE 2010 among 0.3 million applicants
    \item Received \textbf{\textsf{Academic Excellence Award}} for the performance in the term 2010-11.
    \item Ranked \textbf{\textsf{46}} in the \textbf{\textsf{Regional Mathematics Olympiad 2009}} (conducted by SOF).
    \item Awarded CBSE Merit Scholarship for Professional Studies for AIEEE.
    \item Received certificate of merit for exceptional performance in \textbf{\textsf{Maths and Chemistry}} class XII AISSCE.
    \item Selected for the final interview in Mumbai for \textbf{\textsf{Aditya Birla Scholarship Program}}.
    \item Qualified the prelims of \textbf{\textsf{OP Jindal Scholarship Program}}.
    \end{itemize}

    \section{\mysidestyle Internships} 
    \begin{itemize}
    \item \large{\textbf{\textsf{"Partner Profile Management for Facebook Ad Exchange"}}}
      \\ \small{\textit{Summer Internship at Facebook, Menlo Park.(May 2013- July 2013)}}
      \normalsize
      \begin{itemize}
        \item Worked with the Facebook Ads team, specifically on Facebook Exchange (FBX) that allows sophisticated ad partners known as Demand Side Platforms (DSPs) to programmatically buy Ad impressions on Facebook.
        \item Worked with the Facebook Ads team, specifically on Facebook Exchange (FBX) that allows sophisticated ad partners to programmatically buy ad impressions on Facebook.
        \item Understood how the Real Time Bidding (RTB) technology works in online advertising.
        \item Worked on the backend engine including some features like targeting and throttling.
        \item Designed and built an internal tool to efficiently manage the configuration of several partners.
        \item Shipped other features involving change in bid formats for the backend adexchange server. Used C++ for this.
        \item Investigated and implemented a storage layer for the config data.
        \item Worked on moving existing settings from legacy system to the new framework.
          \newpage
        \item Used XHP to create robust UI components and reused standard UI libraries to clean interface.
        \item Studied facebook technologies like generators, HipHop, storage solutions, web server stack, etc. and incorporated them into the tool.
        \item Added useful features like tracking changes at each partner level including ability to revert.
        \item Used Git and Mercurial for version control.
        \item Worked on both frontend and backend engineering.
        \item Modified frontend and backend clients to sync with the latest config changes.
        \item Languages Used: C++, XHP.
        \item Designed and implemented a roll-out plan for moving from old system to the new system.
      \end{itemize}
    \end{itemize}
    %__________________________________________________________________________________________________________________

    \section{\mysidestyle Projects Undertaken}

    \begin{itemize}
    \item \large{\textbf{\textsf{"Path for logarithmic genus graphs/ general graph is in Unambiguous LogSpace Complexity Class"}}}
      \\ \small{\textit{B.tech Project under Prof. Raghunath Tewari (July 2013 - now)}}
      \normalsize
      \begin{itemize}
      \item We are aiming to find a Unambigous LogSpace Class Algorithm for the connectivity problem. Showing this for logarithmic genus graph will make the claim for UL=NL more concrete, showing it for general graph will prove UL=NL. 
      \end{itemize}

    \item \large{\textbf{\textsf{"Justification of Paragraphs in Emacs"}}}
      \\ \small{\textit{Summer Project under Prof. Satyadev NandKumar (May 2012 - June 2012)}}
      \normalsize
      \begin{itemize}
      \item Implemented a modification of Knuth's algorithm for breaking paragraph into lines for mono space characters.
      \item Used the thesis work of Frank Liang for hyphenation of words.
      \item Language Used: Emacs Lisp 
      \end{itemize}
      
    \end{itemize}


    \section{\mysidestyle Course Projects}

    \begin{itemize}

    \item \large{\textbf{\textsf{"Probabilistic model for False recall"}}}
      \\ \small{\textit{Course Project Cognitive Sciences(Feb 2013 - April 2013)}}
      \normalsize
      \begin{itemize}
      \item Used the Deese-Roediger-McDermot(DRM) Paradigm and the fSAM model to build a probabilistic model to predict false recall.
      \item Showed that the results of my model matched those given by fSAM model, and thus strengthen the properties shown by the fSAM model.
      \item Used semantic distances from the Word Association Space(WAS). 
      \item Link: \href{http://home.iitk.ac.in/~rishabhn/se367/project/}{home.iitk.ac.in/$~$rishabhn/se367/project}
      \end{itemize}

    \item \large{\textbf{\textsf{"Online Faculty feedback system"}}}
      \\ \small{\textit{Project in the course Database Management(Feb 2013 - April 2013)}}
      \normalsize
      \begin{itemize}
      \item Used Entity-Relationship modeling, normalization to Boyce Codd Normal Form(BCNF) and managed various integrity constraints in the database design.
      \item Build a frontend in PHP to manage the database, supported authentication of the user and allowed him to give feedback on the registered courses.
      \item Link: \href{https://github.com/rishabhnigam31/Online-Faculty-FeedBack-System}{github.com/rishabhnigam31/Online-Faculty-FeedBack-System}
      \end{itemize}

    \item \large{\textbf{\textsf{"Compilers for a subset of C++ language features"}}}
      \\ \small{\textit{Course Project (Feb 2013 - April 2013)}}
      \normalsize
      \begin{itemize}
      \item Implemented a working compiler for a subset of C++ language.
      \item Worked on lexical, syntax, semantic analysis and code generation for MIPS architecture.
      \item Implemented in C++ and used tool LEX and YACC for building Lexer and Parser.
      \end{itemize}

    \item \large{\textbf{\textsf{"Implementation of Internet Protocol Security over a chat client}}}
      \\ \small{\textit{Course Project in Computer Networks (Sept 2012- November 2012)}}
      \normalsize
      \begin{itemize}
      \item Implemented Secure IP protocols, which included Diffie-Hellman key exchange algorithm, AES encryption
        algorithm and authentication headers using md5 hashing.
      \item Implemented a graphical interface to establish peer to peer connection, send and receive data 
      \end{itemize}

      \newpage

    \item \large{\textbf{\textsf{"Extension of the Pintos Operation System"}}}
      \\ \small{\textit{Course project in Operating Systems (Sept 2012 - November 2012)}}
      \normalsize
      \begin{itemize}
      \item Worked on the pintos operating system to add features such as System Calls, shared memory between different programs, solution to the Readers-Writers problem.
      \item Implemented virtual memory with pure demand paging.
      \item Extended the pintos file-system to include sub-directories, implemented a indexed file system supporting direct, indirect and doubly indirect blocks. 
      \item Added caching in the buffer during read/write operations. 
      \end{itemize}

    \item \large{\textbf{\textsf{"Grammar Acquisition from corpus of sentences"}}}
      \\ \small{\textit{Course Project in Artificial Intelligence (Feb 2012 - April 2012)}}
      \normalsize
      \begin{itemize}
      \item Extended the ADIOS model for Grammar Acquisition, which is based on unsupervised learning, on Hindi Sentences. 
      \item Used this model to generate sentences in Hindi Language.
      \item Concluded that learning from the Hindi Language Corpus, was less accurate then other languages such as English, and thus a more difficult task.
      \item Project Link: \href{http://home.iitk.ac.in/~rishabhn/cs365/projects}{home.iitk.ac.in/~rishabhn/cs365/projects}
      \end{itemize}


    \item \large{\textbf{\textsf{"Implementation of MIPS processor on FPGA unit"}}}
      \\ \small{\textit{Course Project in Computer Organization and Architecture (Feb 2012 - April 2012)}}
      \normalsize
      \begin{itemize}
      \item Used Bluespec Verilog(BSV) to implement a simple processor on FPGA.
      \item Built the Arithmetic and Logical Unit (ALU) and a register file of 32 4-bit registers. The input and output where carried out on FPGA using buttons and 7-segment display.  
      \end{itemize}

    \item \large{\textbf{\textsf{"Learning on the tron bot"}}}
      \\ \small{\textit{Summer Project under Programming Club, IIT Kanpur (May 2011 - June 2011)}}
      \normalsize
      \begin{itemize}
      \item Built an interface using Pygame for the tron game.
      \item Built a learning model to learn by playing against human/bots. Used backtracking to adjust the weights on the feedback of the results.
      \item Tried different Backtracking approaches, and picked the best working one.
      \end{itemize}
      
    \end{itemize}




%______________________________________________________________________________________________________________________
\section{\mysidestyle Relevant Courses} 

\begin{tabular}{@{}p{6cm}p{6.5cm}}
- Randomized Algorithms & - Machine Learning \\
- Introduction to Software Engineering & - Algorithms-II \\
- Computational Complexity &- Compiler Design \\
- Principal of Database Systems &- Introduction to Cognitive Sciences \\
- Principles of Programming Languages &- Computer Networks \\
- Operating Systems &- Theory of Computation \\
- Introduction to Mathematical Logic &- Artificial Intelligence Programming \\
- Programming Tools and Techniques &- Discrete Mathematics \\
- Data Structures and Algorithms & - Introduction to Computer Organization \\
- Fundamentals of Computing &- Mathematics III
\end{tabular}


    %__________________________________________________________________________________________________________________
    % Computer Skills
    \section{\mysidestyle Technical \\Skills} 

    \begin{itemize}
\item \textbf{\textsf{Programming Languages}} - Experienced in C++, C, Python, PHP
\item \textbf{\textsf{Other Programming Languages}} -  MySQL, Oz, Lisp
\item \textbf{\textsf{Other Tools}} - Matlab, Latex, Shell Scripting, gdb
\item \textbf{\textsf{Text Editor}} - Emacs
\item \textbf{\textsf{Version Control}} - Git, Mercurial
\item \textbf{\textsf{Operating Systems}} - Worked on Linux(ubuntu), Mac
\end{itemize}

    %__________________________________________________________________________________________________________________
    % Computer Skills
    \section{\mysidestyle Positions of Responsibility} 

    \begin{itemize}

    \item  \textbf{\textsf{Coordinator, IOPC (International Online Programming Contest), Techkriti'13 August 2012-April 2013}}
      \begin{itemize}
      \item Setup the problems for IOPC, the annual 24 hr long programming contest during Techkriti (\url{www.codechef.com/IOPC2013})
      \item Worked with a team of 5 members on creating and testing the problems for the 24 hr long programming contest(\url{www.codechef.com/IOPC2013})

        \newpage

      \item Achieved a 100\% increase in the number of teams that participated. Total number of teams being 793. 
      \end{itemize}
      
    \item  \textbf{\textsf Coordinator, Software Corner, Techkriti'13 August 2012-April 2013}
      \begin{itemize}
      \item Software Corner represents all the software related competitions during Techkriti. We conducted 5 such events.
      \item Organized the first International High Performance Computing contest, which was hosted on PARAM Yuva supercomputer in collaboration with Centre for Development of Advanced Computing (CDAC)
      \item  Battlecity (an AI design challenge), took the contest online for the first time, got a 12x raise in the number of teams(around 600) participating in the contest. 
      \item Chaos (an Unknown language programming contest), took the contest to international level and successfully managed to conduct the contests on servers from CSE department of IITK for the first time.
      \item Also conducted two onsite events Hack-a-thon and Instant(Puzzle solving and fast coding competition)
      \end{itemize}

    \item \textbf{\textsf Academic Mentor for the course Mathematics-I}
      \begin{itemize}
      \item Planned and took lectures at hostel as well as institute level to help the students understand the fundamentals of Mathematics-1
      \item Mentored the students and provided them with academic assistance.
      \end{itemize}
      

    \end{itemize}

    \section{\mysidestyle Other Activities} 
  
    \begin{itemize}

    \item  \textbf{\textsf Competitive Programming}
      \begin{itemize}
      \item Won $4^{th}$ prize in Instant and $5^{th}$ prize in Battlecity in Techkriti in 2011 and 2012 respectively.
      \item An active participant of programming events in the college, won 1st prize in a coding contest(Kodefest) and
        $2^{nd}$ prize in Blackbox(puzzles and coding contest) in Takneek 2011, an inter-hostel technical event.
      \item Programming Profiles:
        \begin{itemize}
        \item \href{http://codechef.com/users/rishabhnigam31}{Codechef} - short contest rating 82 - long contest rating 1268 in India.
        \item \href{http://spoj.com/users/reincarnated}{SPOJ} - global rank 2483.
        \item \href{http://codeforces.com/profile/rishabhnigam31}{Codeforces} - rating 1644 - India rank 176.
        \item \href{}{Topcoder}
        \end{itemize}
      \end{itemize}

    \end{itemize}



%______________________________________________________________________________________________________________________
\end{resume}
\end{document}


%______________________________________________________________________________________________________________________
% EOF

